\documentclass{article}

\usepackage[pdftex]{graphicx}
\usepackage[czech]{babel}
\usepackage[utf8]{inputenc}
\usepackage{enumitem}
\usepackage{amsmath}
\usepackage{url}
\usepackage{listings}
\usepackage{caption}
\usepackage[usenames,dvipsnames,svgnames,table]{xcolor}

\usepackage[pdftex]{hyperref}
\hypersetup{colorlinks=true,
  unicode=true,
  linkcolor=black,
  citecolor=black,
  urlcolor=black,
  bookmarksopen=true}

\usepackage{xcolor}
\colorlet{mygray}{black!30}
\colorlet{mygreen}{green!60!blue}
\colorlet{mymauve}{red!60!blue}
\lstset{
	backgroundcolor=\color{gray!10},  
	basicstyle=\ttfamily,
	columns=fullflexible,
	breakatwhitespace=false,      
	breaklines=true,                
	captionpos=b,                    
	commentstyle=\color{mygreen}, 
	extendedchars=true,              
	frame=single,                   
	keepspaces=true,             
	keywordstyle=\color{blue},      
	language=c++,                 
	numbers=none,                
	numbersep=5pt,                   
	numberstyle=\tiny\color{blue}, 
	rulecolor=\color{mygray},        
	showspaces=false,               
	showtabs=false,                 
	stepnumber=5,                  
	stringstyle=\color{mymauve},    
	tabsize=3,                      
	title=\lstname                
}

\usepackage[numbers,sort&compress]{natbib}

\newcommand*\justify{
  \fontdimen2\font=0.4em
  \fontdimen3\font=0.2em
  \fontdimen4\font=0.1em
  \fontdimen7\font=0.1em
  \hyphenchar\font=`\-
}

\author{Martin Úbl}

\title{KIV/OS - cvičení č. 9}

\begin{document}

\maketitle

\section{Obsah cvičení}

\begin{itemize}
	\item awaitable soubory (Wait a Notify)
	\item GPIO přerušení
	\item spinlock, mutex, semafor
	\item podmínková proměnná
	\item pojmenovaná roura
	\item EDF plánovač (soft real-time)
\end{itemize}

\section{Úlohy, deadline, awaitable soubory}

V typickém systému reálného času obvykle chceme reagovat na vnější podněty, často z důvodu jeho nasazení na embedded zařízení, které na tyto podněty reaguje příslušnou akcí (která musí být vykonána do určitého času -- deadline). Tyto akce provádí jednotlivé procesy, které jsou tomuto účelu vyhrazené. Ty čekají na příchod podnětu, provedou příslušnou akci,  opět se uspí a to stále dokola opakují. Ve vyšší abstrakci pak lze hovořit o tom, že příchodem podnětu vzniká \emph{task}, který musí být zpracován typicky právě do nějakého shora omezeného času.

\subsection{Model úloh}

Každý systém reálného času může tyto úlohy implementovat různými způsoby. Třeba může na tyto podněty souhrnně čekat jeden \uv{rodičovský} proces, a ten po příchodu vytvoří proces specifický, který danou úlohu zpracuje. Vzhledem k tomu, že primárním požadavkem jakéhokoliv systému reálného času je minimální odezva, může toto řešení být nevyhovující. Prakticky proto můžeme procesy pro každou úlohu vytvořit rovnou a každý individuálně nechat čekat na daný podnět. V našem systému zvolíme tento druhý model.

V momentě, kdy podnět přijde, je třeba nastavit deadline. Deadline, tedy horní omezení času zpracování daného podnětu, je třeba nastavit v momentě, kdy task vzniká. Prakticky v námi zvoleném modelu jde o moment probuzení procesu.

Deadline je důležitým parametrem plánování procesu -- o plánování procesů na základě deadline více pojednává jedna z posledních kapitol.

\subsection{Rozhraní}

Zmínili jsme, že proces bude čekat na podnět -- zbývá tedy jen vyřešit konkrétní způsob, jakým se bude proces uspávat, a jakým bude probuzen. V kontextu našeho systému se nabízí pro tento mechanismus čekání využít souborový systém, k němuž jsme již v minulosti vytvořili rozhraní v podobě systémových volání \texttt{open}, \texttt{close}, \texttt{read} a \texttt{write}. To by ve své podstatě stačilo i pro tyto účely za předpokladu, že dovolíme \texttt{read} blokovat. Pro lepší definici sémantiky však dodefinujme dvě další operace -- \texttt{wait} a \texttt{notify}.

Operace \texttt{wait} uspí proces nad souborem, pokud to bude nutné. Sémanticky tedy bude proces čekat na zdroj, až bude dostupný. Pokud zdroj dostupný je, proces blokovat (uspávat) nebude.

Operace \texttt{notify} notifikuje čekající procesy nad souborem, pokud nějaké jsou. Sémantika bude závislá na druhu souboru -- buď může jít čistě o notifikaci procesu (např. podmínková proměnná), nebo může jít o přidání zdrojů (např. roura).

% awaitable soubor
% wait + notify do IFile
% wait + notify syscall (s deadline)
% sleep (s deadline) --> periodicke tasky

% GPIO preruseni (ioctl) --> rozsireni GPIO driveru a GPIO file

% spinlock
% mutex + filesystem
% semafor + filesystem
% podminkova promenna + filesystem
% pojmenovana roura + filesystem

% EDF planovac

% power management - idle proces a WFE



\end{document}























