\documentclass{article}

\usepackage[pdftex]{graphicx}
\usepackage[czech]{babel}
\usepackage[utf8]{inputenc}
\usepackage{enumitem}
\usepackage{amsmath}
\usepackage{url}
\usepackage{listings}
\usepackage{caption}
\usepackage[usenames,dvipsnames,svgnames,table]{xcolor}

\usepackage[pdftex]{hyperref}
\hypersetup{colorlinks=true,
  unicode=true,
  linkcolor=black,
  citecolor=black,
  urlcolor=black,
  bookmarksopen=true}

\usepackage[numbers,sort&compress]{natbib}

\newcommand*\justify{
  \fontdimen2\font=0.4em
  \fontdimen3\font=0.2em
  \fontdimen4\font=0.1em
  \fontdimen7\font=0.1em
  \hyphenchar\font=`\-
}

\author{Martin Úbl}

\title{KIV/OS - cvičení č. 5}

\begin{document}

\maketitle



\section{Obsah cvičení}

\begin{itemize}
	\item alokátor paměti
	\item kernelová halda
	\item preemptivní round-robin plánovač
\end{itemize}

\section{Alokátor paměti}

Alokace paměti je obecně poměrně složitý problém, pokud to chceme udělat správně a co nejefektivněji. My implementujeme v rámci cvičení takový alokátor, který bude pouze velmi primitivní. Rozhraní však bude mít finální a lepší implementace pak bude ponechána na další cvičení.

Nyní náš operační systém nezná např. ani stránkování -- to přibyde až v některém z dalších cvičení. Alokátor proto moc optimalizovat nebudeme, jelikož schéma alokace pak bude určitě odlišné. Mohli bychom ale už trochu počítat s tím, že budeme chtít alokovat stránky o nějaké konkrétní velikosti. Rovněž rovnou počítejme s tím, že oblast, ve které chceme alokovat, bude nějak zdola a shora omezena.

Definujme si proto soubor \texttt{memmap.h}, kde si tyto meze stanovíme. Oblast začátku můžeme zvolit v podstatě libovolně tak, aby se nepřekrývala s kódem (a daty) jádra. Pro teď si zvolme nějaký počátek ručně, v budoucnu však můžeme například použít již známý trik -- v linker skriptu definovat zarážku za poslední sekcí a tu si vyzvednout. Pak je třeba ji zarovnat na nejbližší vyšší násobek velikosti stránky a lze ji použít.

Horní mez stanovme dle možností architektury a desky. V manuálu se dočteme, že nejbližší horní hranicí je memory-mapped IO region. Pak můžeme horní mez nastavit prostě na \texttt{Peripheral\_Base}. Pozor ale -- naše zařízení zdaleka tolik fyzické paměti nemá. Řešením by mohlo být například vyzvednutí skutečného rozsahu paměti pomocí mailbox mechanismu. To ale není obsahem tohoto cvičení.
\newpage
Obsah souboru \texttt{memmap.h} pak může vypadat třeba takto:
\begin{lstlisting}
namespace mem
{
  constexpr uint32_t LowMemory = 0x20000;
  constexpr uint32_t HighMemory = Peripheral_Base;
  constexpr uint32_t PageSize = 0x4000;
  constexpr uint32_t PagingMemorySize =
                           HighMemory - LowMemory;
  constexpr uint32_t PageCount =
                      PagingMemorySize / PageSize;
}
\end{lstlisting}

Pokud to ještě v tento moment není zřejmé, operujeme v módu, kdy je každý kus paměti přímo přístupný a jeho adresa není nijak překládána. Pokud se tedy jádro v jakýkoliv moment rozhodne, že chce něco zapsat do paměti na adrese \texttt{0x00123456}, tak se to prostě s největší pravděpodobností povede.

Implementací alokátoru vlastně všechnu dostupnou paměť \uv{přidělíme} právě jádru, a alokátor bude tuto paměť spravovat. Měl by podporovat jak alokaci, tak i dealokaci bloku paměti. Z podstaty věci ale samotný alokátor neví, komu paměť přiděluje -- to pak je otázkou příslušného \uv{správce}. Tím je třeba správce haldy jádra, správce procesů a tak dále. Tento správce zodpovídá jak za \uv{úměrnou} alokaci, tak za správnou a úplnou dealokaci. Rovněž může dále vnitřně paměť rozdělovat na menší kousky -- to typicky budeme chtít jak u haldy jádra, tak i u procesů. K tomu ale zase později.

Jednou z nejjednodušších implementací alokátorů stránek (resp. později rámců) je bitová mapa. V té každý bit reprezentuje jednu stránku (rámec) paměti, přičemž hodnota \uv{vypnuto} znamená, že je blok volný a \uv{zapnuto}, že byl někomu přidělen. Jelikož víme, kolik stránek můžeme maximálně alokovat (\texttt{PageCount}), tak také víme, jak velkou bitovou mapu budeme potřebovat.

Uveďme příklad. Za předpokladu, že je paměť velká 512 MiB ($512*1024*1024$ B) a stránka (rámec) má velikost \texttt{0x4000} (16 kiB, $16*1024$ B), pak tato bitová mapa musí čítat 32768 bitů (4096 B, tedy 4 kiB). Tento výpočet lze provést v čase překladu přímo z konstant. Obětování 4 kiB z paměti pro bitmapu, která reprezentuje 512 MiB paměťový prostor není zase tolik. Jde to ale samozřejmě trochu lépe a efektivněji -- o tom více pojednávají přednášky KIV/OS nebo KIV/ZOS. Pro jednoduchost tady na cvičení implementujme bitovou mapu s algoritmem typu \emph{first-fit}, tedy $O(n)$ alokátorem, který hledá první volný blok od začátku alokovatelné paměti.

Mějme tedy třídu \texttt{CPage\_Manager}, která bude naším alokátorem. Ta bude obsahovat metodu \texttt{Alloc\_Page()}, která bude alokovat novou stránku a bude vracet adresu prvního bajtu (nebo 0, pokud žádná stránka není volná). Dále bude obsahovat metodu \texttt{Free\_Page(uint32\_t addr)}, která bude přejímat adresu prvního bajtu stránky k uvolnění jako parametr a stránku bude ve své implementaci uvolňovat.

Implementujeme bitmapový \emph{first-fit}, a tedy potřebujeme mít alokovanou bitmapu. Ve třídě proto vytvořme atribut, který bude představovat bitovou mapu alokovatelné paměti \texttt{uint8\_t mPage\_Bitmap[mem::PageCount / 8];}.

Samotná implementace pak musí obsahovat inicializaci (zde nejlépe asi konstruktorem), ve které označíme všechny bloky za volné:
\begin{lstlisting}
CPage_Manager::CPage_Manager()
{
  for (int i = 0; i < sizeof(mPage_Bitmap); i++)
    mPage_Bitmap[i] = 0;
}
\end{lstlisting}
Alokace pak probíhá snadno -- nalezením prvního oktetu stránek, který není celý alokovaný (abychom to trochu urychlili), v něm nalezení volné stránky a následným označením a vrácením:
\begin{lstlisting}
uint32_t CPage_Manager::Alloc_Page()
{
  uint32_t i, j;
	
  for (i = 0; i < mem::PageCount; i++)
  {
    if (mPage_Bitmap[i] != 0xFF)
    {
      for (j = 0; j < 8; j++)
      {
        if ((mPage_Bitmap[i] & (1 << j)) == 0)
        {
          const uint32_t page_idx = i*8 + j;
          mPage_Bitmap[page_idx / 8] |= 1 << (page_idx % 8);
          return mem::LowMemory + page_idx * mem::PageSize;
        }
      }
    }
  }

  return 0;
}
\end{lstlisting}
Uvolnění je pak již snadné -- stačí příslušný bit vynulovat:
\begin{lstlisting}
void CPage_Manager::Free_Page(uint32_t addr)
{
  const uint32_t page_idx = addr / mem::PageSize;
  mPage_Bitmap[page_idx / 8] &= ~(1 << (page_idx % 8));
}
\end{lstlisting}
V systémech, kde je nutné dbát na bezpečnost (typicky spíše interaktivní systémy) je možné při dealokaci ještě systémově přemazat obsah dealokované paměti (nějakým vzorem nebo náhodnými daty) -- pokud by k tomu nedošlo, ve stránce byla nějaká citlivá data (hesla, šifrovací klíče, ...) a stránku si pak alokoval nějaký jiný proces, k datům by pak měl přístup.

\section{Kernelová halda}

Nyní již máme způsob, jakým přidělovat stránky (rámce) paměti. Jádro ale potřebuje často alokovat malé kousky paměti pro relativně drobné struktury, a proto potřebujeme mít možnost stránky rozmělnit na drobnější kousky.

...

\end{document}























